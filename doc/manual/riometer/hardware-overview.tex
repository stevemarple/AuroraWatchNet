\chapter{Overview of the hardware}

\section{Introduction}

The riometer data logger is an extension of the AuroraWatchNet
project, an open-source magnetometer designed for generating real-time
alerts when aurora might be visible by lower-latitude observers. The
magnetometer was originally designed with a battery-powered remote
sensor unit that was located outdoors and away from human
disturbance. It connected via a radio link to a base unit indoors that
contained a Raspberry Pi single board computer. Later a power over
Ethernet (\PoE) remote sensor unit was designed. The hardware and
software have had extensive testing and operation since the project's
inception in 2012.

In the riometer data logger all of the electronics are located
indoors. For convenience the sensor unit and Raspberry Pi are
contained in the same enclosure. For single-beam riometer systems the
riometer is also contained in the same enclosure. For imaging riometer
systems the riometers and Butler matrix units are contained in their
own enclosures.

\section{Hardware description}

The sensor unit of the riometer data logger makes uses of the existing
Power over Ethernet microcontroller board. It is itself a derivation
of the Calunium project by the author to create an Atmel ATmega1284P
Arduino clone. The benefit of the ATmega1284P is the larger memory
available compared to most Arduino boards, yet still retains the
convenience of a through-hole package for easy home assembly and
tool-less replacement during servicing.

All of the data acquisition, timing, temporary storage, collection of
house-keeping data and transmission over Ethernet is performed by an
Atmel ATmega1284P microcontroller running at \MHz{20}. The firmware
was developed using the Arduino development environment. This was
chosen at the beginning of the AuroraWatchNet project to foster an
open environment conducive to collaboration. Many of the software
modules used to support the data logger were written specifically for
the project and have been released as independent open source
software. Most of the software modules implement their own state
machines to allow efficient and cooperative real time
operation. Remote over the air firmware updates are made possible by
the xboot bootloader; updates via the \usb\ interface are also
supported.

Analogue signals are converted to digital values by one or more
Microchip MCP3424 analogue to digital converters (one for wide beam
riometer systems, for imaging riometers each image column has its own
\adc). In the case of multiple \adc s they are simultaneously
commanded to being sampling. The microcontroller is responsible for
initiating data sampling, using its internal software real time clock
or a \gnss\ pulse-per-second signal. If over-sampling has been
selected the samples are reduced to a single value for each beam. The
resulting samples for each beam are stored in a buffer for subsequent
transmission over Ethernet to a Python data logger daemon running on a
Raspberry Pi.

The Ethernet interface uses a standard Arduino Ethernet shield. Either
the older Wiznet W5100 model or the newer Wiznet W5500 model can be
used but the correct firmware version must be programmed into the
microcontroller.


\section{Clock sources}

The sensor unit has a number of clock sources available and will use
the best source available to timestamp the data. The timestamp
corresponds to the beginning of the data sampling period (not
necessarily the exact moment data sampling began as there may be a
pre-sample delay configured). The clock sources are:
\begin{itemize}
\item The onboard real time clock \ic. The clock runs from a
  \Hz{32768} watch crystal with a typical accuracy of
  \SI{20}{ppm}. The time can only be read back with a precision of one
  second.
\item The \gnss\ combined clock source. When multiple satellite
  constellations are in use the \gnss\ module derives its time and
  location fix from all of the suitable satellites in view. Clock
  accuracy is excellent, the pulse-per-second output has an accuracy
  of \SI{\pm15}{\nano\second} (requires the correct antenna delay
  compensation for this accuracy to be achieved). Requires the
  firmware to be compiled with \code{FEATURE_GNSS} enabled.
\item The server clock, obtained from the acknowledgement message. The
  server clock time is sent only when the Raspberry Pi \ntp\ service
  has a valid time.
\end{itemize}

To manage these different clock sources the microcontroller implements
a software real time clock module. On start up the clock is
initialized from the onboard real time clock \ic, with the limitation
that the precision is limited to one second. However the software
clock uses the \Hz{32768} square wave output to achieve higher
internal resolution.

When the server clock time is available it is compared to the internal
software clock, and if necessary the software clock is adjusted. Time
adjustments are limited to when the difference is deemed too large, to
avoid adding unnecessary clock jitter to the acquired data samples.

When the \gnss\ module has a valid position and time fix its next
pulse-per-second output is used to set the internal software
clock. Data acquisition commences immediately on receipt of a valid
pulse-per-second edge, or if one is not available at the next
prescribed sampling time using the internal software clock. Thus when
the \gnss\ module has a valid fix data timestamps are placed at the
second boundary, otherwise they will occur (to best effort) at $n$
second intervals where $n$ is the sampling interval in seconds but not
necessarily on a second boundary.

The hardware real time clock \ic\ is set periodically to ensure
accurate time-keeping from a cold-start when the Raspberry Pi does not
have a valid time from \ntp\ and before the \gnss\ module has been
able to acquire a position and time fix.

\helpbox{%
  The server time clock is never used for time-keeping whilst the
  \gnss\ module has a valid position and time fix.
\item The \gnss\ position, time and satellites in view information are
  reported with the next data sample. Thus when monitoring the serial
  console or \gnss\ data files it may appear that this information is
  sent late. The microcontroller firmware uses the current \gnss\ time
  to compute the time for the next pulse-per-per second signal to
  ensure the correct data stamps are applied to the data. This can be
  confirmed by monitoring a \gnss\ clock or the time at
  \url{https://time.is/} and comparing with the incoming data
  timestamps.}
