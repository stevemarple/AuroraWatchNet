\chapter{Overview of the hardware}

\section{Introduction}
The magnetometer is designed for low-power operation, simple
installation and ease of construction. The entire design is open
source, allowing anyone with reasonable soldering ability to construct
one.

The magnetometer has two major parts, the base unit and the sensor
unit (\figurename~\ref{fig:system-overview}). The sensor unit is located
outdoors, away from buildings, cars and other sources of human
disturbance. It is battery powered and communicates with the base unit
by a radio link (\MHz{433} or \MHz{868}), enabling the sensor to be
installed without any wiring to the base unit. The base unit is placed
indoors and should be positioned such that there are the minimum
number of walls between it and the sensor unit.

\begin{figure}
  \centering
  \includegraphics[keepaspectratio,width=\textwidth]{%
    calunium-mag/images/system-overview}
  \caption[System overview]%
  {System overview.}
  \label{fig:system-overview}
\end{figure}

\section{Sensor unit}

The sensor unit (figure~\ref{fig:sensor-unit}) is contained inside a
waterproof enclosure approximately \SI{1.1}{\metre} high which is
partially buried to reduce temperature variations and to provide a
stable foundation. The sensor itself is placed at the bottom of the
enclosure, approximately \SI{0.85}{\metre} below ground. The
microcontroller, radio module and battery are positioned in the top
part of the enclosure, above ground level. Insulating material (\eg\
rockwool) is used to fill the space in-between.

The
\href{http://blog.stevemarple.co.uk/search/label/Calunium}{Calunium}
microcontroller board is based on the popular
\href{http://arduino.cc}{Arduino} platform but uses the more powerful
Atmel ATmega1284P microcontroller.

\begin{figure}[!h]
  \centering
  \includegraphics[keepaspectratio,height=10cm]{images/sensor-unit}
  \caption[Sensor unit]{%
    Sensor unit. \photoCredit{Steve Marple}{\ccBySaTwo}{%
      http://www.flickr.com/photos/stevemarple/8499204572/} 
  }
  \label{fig:sensor-unit}
\end{figure}

\begin{figure}
  \centering
  \includegraphics[keepaspectratio,width=\textwidth]{%
    images/calunium-v2-1}
  \caption[Calunium microcontroller PCB]{%
    Calunium microcontroller \pcb. %
    \photoCredit{Steve Marple}{\ccBySaTwo}{%
      http://www.flickr.com/photos/stevemarple/10786865096/}}
  \label{fig:calunium}
\end{figure}

\begin{figure}
  \centering
  \includegraphics[keepaspectratio,width=\textwidth]{%
    images/flc100-shield}
  \caption[FLC100 shield]{The FLC100 shield fits onto the Calunium
    microcontroller \pcb. \photoCredit{Steve Marple}{\ccBySaTwo}{%
      http://www.flickr.com/photos/stevemarple/10787109594/}}
  \label{fig:flc-100-shield}
\end{figure}

\begin{figure}
  \centering
  \includegraphics[keepaspectratio,width=\textwidth]{%
    images/calunium-flc100-shield.jpg}
  \caption[Calunium microcontroller board and FCL100 shield]{%
    Calunium microcontroller board and FCL100 shield. 
    \photoCredit{Steve Marple}{\ccBySaTwo}{%
      http://www.flickr.com/photos/stevemarple/10913562526/}}
  \label{fig:calunium-flc-100-shield}
\end{figure}

\clearpage
\section{Base unit}

The base unit is a \href{http://www.raspberrypi.org}{Raspberry Pi}
single-board computer with a radio transceiver unit. The Ethernet
interface of the Raspberry Pi is used to send the magnetic field
measurements to AuroraWatch UK. When the Raspberry Pi is accessed over
the network with Secure Shell (\ssh) a display and keyboard are not
needed. The Raspberry Pi runs the Raspbian linux distribution. The
receiving software is written in Python.

\begin{figure}[!h]
  \centering
  \includegraphics[keepaspectratio,height=10cm]{images/base-unit}
  \caption[Base unit]{%
    Base unit. \photoCredit{Steve Marple}{\ccBySaTwo}{%
      http://www.flickr.com/photos/stevemarple/10787215844/}}
  \label{fig:base-unit}
\end{figure}
