\chapter{Site requirements}

\section{Sensor unit requirements}
The sensor unit should be located outside away from human
distrubances. The site for the sensor unit should be chosen with
regard to the following requirements, with the highest priority given
first.

\begin{itemize}
\item Within range of the base unit. For systems using radio
  communication this is likely to be less than \SI{50}{\metre}. For
  systems using \PoE\ the distance is limited by the maximum Ethernet
  segment length, \SI{100}{\metre}.
\item Away from moving metal objects, for example, trains (more than
  \SI{50}{\metre}), cars (more than \SI{20}{\metre}) and garage doors.
\item Away from static metal objects, in particular those containing
  the \emph{ferro-magnetic} materials iron, nickel and cobalt.
\end{itemize}

The FLC100 fluxgate magnetometer sensor is slightly sensitive to
temperature variations. To ensure correct behaviour a stabilised
temperature environment is required. This is obtained by burying the
sensor.

\section{Base unit requirements}

The Raspberry Pi requires mains power and a wired network
connection. Wireless networking is also possible but has not been
tested; follow the instructions on the Raspberry Pi website. The base
unit should be located indoors. Systems using radio communication will
benefit from the base unit being located as close to the sensor unit
as possible.

\subsection{Network requirements}

The following network requirements are needed for the system to
operate fully:
\begin{itemize}
\item \dns\ resolution. This is normally provided
  as standard on most networks.
\item Outgoing access on port 80 (\http) and 443 (\https). Required
  for software updates and data transfer to AuroraWatch UK and the Met
  Office Weather Observations Website.
\item Outgoing access on port 123 (\ntp), or access to a local \ntp\
  server.
\item Outgoing \ssh\ access (port 22) is required if \code{rsync} or
  \code{rrsync} data transfers are enabled.
\end{itemize}




