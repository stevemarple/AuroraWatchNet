\chapter{Installation procedure}

\subsection{Tools required}

\begin{itemize}
\item Spade.
\item Fork.
\item Small bucket or other container to remove soil from the bottom
  of the hole.
\item Compass.
\end{itemize}

The following items are optional but if they are available they may be
useful for digging the hole.
\begin{itemize}
\item Soil auger.
\item Post hole digger.
\end{itemize}

\section{Network infrastructure}
\todo

\section{Base unit installation}

Connect the Raspberry Pi to wired ethernet connection. Connect the
keyboard, mouse and monitor (if using) before powering up the
Raspberry Pi. Connect the Raspberry Pi to a \volt{5} power supply with
an output current rating of at least \SI{700}{\milli A}. If you are
not using a monitor connect to the Raspberry Pi using \ssh, the
default hostname is \filename{raspberry.local}.


\section{Installing the sensor unit}

The sensor unit must be should be buried to a depth of
\SI{0.85}{\metre}, if it is buried too shallow the unit will be more
susceptible to temperature variations. After digging a suitable hole install the enclosure as vertical as
possible and backfill the hole. Insert the wooden frame and rockwool
into the enclosure. Using a compass align the north arrow on the
wooden frame to point towards magnetic north. \todo[Complete\ldots]





