\chapter{Site requirements}

\section{Sensor requirements}
The sensor should be located outside away from human distrubances. The
site for the sensor should be chosen with regard to the following
requirements, with the highest priority given first.

\begin{itemize}
\item Away from moving metal objects, for example, trains (more than
  \SI{50}{\metre}), cars (more than \SI{20}{\metre}) and garage doors.
\item Away from static metal objects, in particular those containing
  the \emph{ferro-magnetic} materials iron, nickel and cobalt.
\end{itemize}

The FLC100 fluxgate magnetometer sensor is slightly sensitive to
temperature variations. To ensure correct behaviour a stabilised
temperature environment is required. This is obtained by burying the
sensor.

\section{Raspberry Pi requirements}

The Raspberry Pi requires power and a network communication. The
normal mode of operation is to utilise a \ieee\ 802.3af standard \PoE\
network. Alternatives such as passive \PoE\ are possible but not
recommended.

\subsection{Network requirements}

The following network requirements are needed for the system to
operate fully:
\begin{itemize}
\item \dns\ resolution. This is normally provided
  as standard on most networks.
\item Outgoing access on port 80 (\http) and 443 (\https). Required
  for software updates and data transfer to AuroraWatch UK and the Met
  Office Weather Observations Website.
\item Outgoing access on port 123 (\ntp), or access to a local \ntp\
  server.
\item Outgoing \ssh\ access (port 22) is required if \code{rsync} or
  \code{rrsync} data transfers are enabled.
\end{itemize}




