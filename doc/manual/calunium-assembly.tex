\chapter{Calunium assembly}

The Calunium microcontroller development board is intended to be a
flexible system for both development and embedded use. As such it has
various hardware options and careful attention must be paid to
assembly it for optimum performance. Parts which are not needed are
omitted to lower power consumption (\eg, power LED, USB controller).

\section{Calunium version 2.0}


\subsection{Parts list}

Omit the parts relating to the \volt{5} regulator, \ldots.

For the I2C bus use \kohm{4.7} pull-up resistors.


\begin{landscape}
  \begin{figure}[p]
    \centering
    \includegraphics[keepaspectratio,width=28cm,height=16cm]{%
      ../../hardware/Calunium/hardware/pcb/Calunium_v2/Calunium_v2_sch}  
    \caption{Calunium v.~2.0 circuit diagram.}
    \label{fig:calunium-v2.0-cct-diag}
  \end{figure}
\end{landscape}



\section{Calunium version 2.1}

\subsection{Parts list}

Omit the parts relating to the \volt{5} regulator, \ldots.


For battery operation omit the parts relating to the USB interface as
the MCP2200 consumes too much power. Also omit the power LED and its
current-limiting resitor (R1?).


Fit a \ohm{680} current-limiting resitor for the debug LED.

\begin{landscape}
  \begin{figure}[p]
    \centering
    \includegraphics[keepaspectratio,width=28cm,height=16cm]{%
      {../../hardware/Calunium/hardware/pcb/Calunium_v2.1/Calunium_v2.1_sch}.pdf}  
    \caption{Calunium v.~2.1 circuit diagram.}
    \label{fig:calunium-v2.1-cct-diag}
  \end{figure}
\end{landscape}

\section{Programming the firmware}
