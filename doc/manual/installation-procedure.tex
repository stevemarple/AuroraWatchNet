\chapter{Installation procedure}

\subsection{Tools required}

\begin{itemize}
\item Spade.
\item Fork.
\item Small bucket or other container to remove soil from the bottom
  of the hole.
\item Compass.
\end{itemize}

The following items are optional but if they are available they may be
useful for digging the hole.
\begin{itemize}
\item Soil auger.
\item Post hole digger.
\end{itemize}

\section{Base unit installation}

Connect the Raspberry Pi to wired ethernet connection. Connect the
keyboard, mouse and monitor (if using) before powering up the
Raspberry Pi. Connect the Raspberry Pi to a \volt{5} power supply with
an output current rating of at least \SI{700}{\milli A}. If you are
not using a monitor connect to the Raspberry Pi using \ssh, the
default hostname is \filename{raspberry.local}.


\section{Determining the maximum range for the radio link}
\label{sec:find-maximum-range}
Before installing the sensor unit first check the maximum range for
radio communication. Note that this will vary according to many
conditions and be sure to position the sensor well within the maximum
range.

The maximum range can be determined with the following procedure:
\begin{enumerate}
\item Ensure that the Raspberry Pi is turned on and restart the
  recording daemon (p.~\pageref{awnetd-restart}). Monitor the recording
  process (p.~\pageref{monitor-daemon-output}).
\item Disconnect the remote sensor \pcb\ from the FLC100 shield. Place
  the remaining circuitry about \SI{5}{\metre} away from the Raspberry
  Pi. Keep the two units at least \SI{2}{\metre} apart at all times to
  prevent overloading the radio receivers.
\item Connect the battery. The green \led\ on the Calunium \pcb\ will
  flash 3 times with a frequency of about \Hz{1} to indicate that the
  bootloader is waiting for data. The \led\ will then turn off.
\item \label{item-led-flash} About a second later the \led\ should
  light again, indicating the start of data transfer to the Raspberry
  Pi. If an acknowledgement is received the \led\ will turn off. This
  process should happen quickly, so that the \led\ flashes on
  momentarily. Just after the \led\ turns on the Raspberry Pi should
  output the received data message, along with its response to the
  sensor unit. If the \led\ remains on it indicates the
  microcontroller has not received an acknowledgement; see
  section~\ref{debugging-communication-errors}.
  
  This step repeats itself following every data sampling operation,
  normally every \SI{30}{\second}. The time interval can be reduced by
  altering the sampling period
  (section~\ref{sec-sampling-interval}). A functioning radio
  communication link is required to alter the sampling period.
\item Move the sensor unit to its intended location. Try to position
  it at the same height as it will be when buried, approximately
  \SI{10}{\centi\metre} above ground. Keep the radio antenna
  vertical. If the \led\ continues to flash briefly every
  \SI{30}{\second} the radio link is operating correctly. If not then
  find another location.
\item If the radio link appears to work correctly at the desired
  location then aim to find the maximum range by steadily increasing
  the distance between the sensor unit and base unit. Remember that the
  link is only tested after each sampling period. If the \led\ remains
  on move the sensor unit closer to the base unit and wait for the
  link to re-establish, indicated by the \led\ switching off.
\end{enumerate}

\helpbox{To save power the \led\ indicates the link status only for
  approximately 15 minutes after the power has been connected or the
  reset button pressed. If the \led\ has stopped indicating the link
  status press the reset button on the Calunium \pcb.}

\subsection{Factors which alter the maximum range}
The following factors can influence the range of the radio link:
\begin{itemize}
\item Obstructions. Try to find locations for the sensor unit and base
  unit which have direct line of sight. It is acceptable for walls to
  be in the path but note they will attenuate the signal. Avoid raised
  ground whenever possible. Placing the base unit on the first or
  second floor is likely to give improved range.
\item Sources of radio noise. Some electrical equipment may cause
  \rfi, which will reduce the range of the radio link. Try to keep
  both the sensor unit and base unit away from other electrical
  equipment, particularly those containing radio
  transmitters. ``Smart'' electric meters can contain radio
  transmitters which operate on the same \MHz{868} \ism\ radio band.
\end{itemize}


\section{Installing the sensor unit}

The sensor unit must be should be buried to a depth of
\SI{0.85}{\metre}. If it is buried deeper then the radio antenna may
be too close to the ground and compromise the wireless communication
range, if it is buried too shallow the unit will be more susceptible
to temperature variations.

After digging a suitable hole install the enclosure as vertical as
possible and backfill the hole. Insert the wooden frame and rockwool
into the enclosure. Using a compass align the north arrow on the
wooden frame to point towards magnetic north. Connect the RJ45 cable
to the FLC100 shield. Connect the power lead. Fit two D cell batteries
into the battery holder and place on the wooden frame. Press the reset
button and observe the green \led. It should flash 3 times to indicate
the bootloader operation and then intermittently to indcate
successful data transmission; see section~\ref{sec:find-maximum-range}.





