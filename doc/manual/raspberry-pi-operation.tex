\chapter{Raspberry Pi operation}

\section{Introduction}

For generic operation of the Raspberry Pi (setting the hostname,
assigning a fixed \ip\ address \etc) please see the Raspbian
documentation,
\url{http://www.raspbian.org/RaspbianDocumentation}.

\section{Shutting down the Raspberry Pi}

The Raspberry Pi must be shutdown cleanly before power is removed:
\begin{Cmd}
sudo shutdown -h now
\end{Cmd}
Before removing the power wait until only the red power \led\ is lit;
wait a further two seconds to ensure futher access to the \sd\ card
are not needed. If the power is removed whilst data is being written
to the \sd\ will corrupt it file system.


To reboot the Raspberry Pi use
\begin{Cmd}
sudo shutdown -r now
\end{Cmd}

\section{Starting and stopping the data recording daemon}

Data is recorded on the Raspberry Pi using a \emph{daemon} process,
which is started and stopped by the Debian init scripts. The scripts
must be started and stopped as user \rootUser, the actual data
recording process runs as user \piUser.

To start data recording
\begin{Cmd}
sudo /etc/init.d/awnetd start
\end{Cmd}

To stop data recording
\begin{Cmd}
sudo /etc/init.d/awnetd stop
\end{Cmd}

It is also possible to check the status of the data recording process
\begin{Cmd}
sudo /etc/init.d/awnetd status
\end{Cmd}

\label{awnetd-restart}
The \code{restart} option forcibly stops recording (if running) and
then starts it again:
\begin{Cmd}
sudo /etc/init.d/awnetd restart
\end{Cmd}

\section{Monitoring the data recording process}

The data recording process directs its standard output and error
streams to a virtual terminal using \code{screen}. It is possible to
attach to this virtual terminal to monitor the output.

As user \piUser
\begin{Cmd}
screen -r awnet
\end{Cmd}

To exit from \code{screen} type \keystroke{CTRL}-\keystroke{a},
\keystroke{d}. 

\warningbox{Pressing \keystroke{CTRL}-\keystroke{c} will terminate the recording
  process.}



