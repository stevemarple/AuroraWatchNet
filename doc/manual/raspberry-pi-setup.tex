\chapter{Raspberry Pi setup}

\section{SD card creation}

Download the latest Raspbian image and copy to the SD card following
the instructions on the Raspberry Pi web site. \textbf{You cannot copy
  the compressed image to a FAT partition on the SD card}.

\section{Configuring Raspbian}

Raspbian is most easily configured by booting the new image. If you
are able to discover the IP address (for instance, by checking the
DHCP tables of your home router) you can do this over the network
using SSH. Otherwise you must use attach a keyboard and monitor to the
Raspberry Pi. If you are familiar with linux it is also possible to
edit the files by mounting the SD card on another Linux system.

\subsection{/dev/ttyAMA0 serial port setup}

Disable the console and \code{getty} process from running on
\filename{/dev/ttyAMA0}. Edit \filename{/boot/cmdline.txt} to remove
the parts which relate to \filename{ttyAMA0}. Remove
\begin{Cmd}
console=ttyAMA0,115200 kgdboc=ttyAMA0,115200
\end{Cmd}


Edit \filename{/etc/inittab}. Find the line relating to
\filename{ttyAMA0}. Either delete the line entirely or comment it out
by inserting a hash character (\#) at the start of the line.

\subsection{Raspbian configuration}

Log in as \piUser\ and run
\begin{Cmd}
sudo raspi-config
\end{Cmd}

\subsubsection{Change user password}
\textbf{If the default password has not been changed then do so now to keep
your system secure.}

\subsubsection{Internationalisation options}
Select \code{Internationalisation Options} and then
\code{Change Timezone}. For geographic area select %
\code{None of the above}, then select \code{UTC}. Select \code{OK}.

\subsubsection{Advanced options}
Select \code{Advanced Options} and then \code{Memory
  Split}. Set the GPU memory to \code{16} (MB).

\subsubsection{Expand Filesystem}
Finally select \code{Expand Filesystem}. Although the first option
do this last. Choose \code{Finish} and then reboot.

\section{Install missing software packages}
As user \rootUser
\begin{Cmd}
apt-get install screen lsof
\end{Cmd}

\section{Installing the AuroraWatchNet server software}

\subsection{Install the Git repository}
As \piUser
\begin{Cmd}
git clone --recursive git://github.com/stevemarple/AuroraWatchNet.git
mkdir \mytilde/bin
cd \mytilde/bin
ln -s ../AuroraWatchNet/software/server/awnetd/awnetd.py
ln -s ../AuroraWatchNet/software/server/bin/log_ip
\end{Cmd}

\subsection{Configure \protect\filename{cron}}

As user \piUser
\begin{Cmd}
crontab -e
\end{Cmd}

In the \filename{nano} editor add the following lines: \todo[Add lines
for data transfer]
\begin{Cmd}
@reboot /home/pi/bin/log_ip reboot > /dev/null 2>&1
@hourly /home/pi/bin/log_ip > /dev/null 2>&1
\end{Cmd}
Save the file, \keystroke{CTRL}-\keystroke{x}, \keystroke{y},
\myreturn.

\subsection{Create configuration file}

As user \rootUser
\begin{Cmd}
mkdir /data
chown pi.pi /data
nano /etc/awnet.ini
\end{Cmd}

\todo[Create file contents, perhaps using a template copied from
the repository]

\subsection{Create init file for server daemon}
As user \rootUser
\begin{Cmd}
cd /etc/init.d
ln -s /home/pi/AuroraWatchNet/software/server/awnetd/awnetd.sh awnetd
update-rc.d  awnetd defaults
\end{Cmd}

