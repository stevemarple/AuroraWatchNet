\chapter{Contributing data to AuroraWatch UK}

\section{Use of data by AuroraWatch UK }

Data contributed to AuroraWatch UK will be combined with other
magnetometer data for the purpose of generating AuroraWatch UK or
other auroral-related alerts. In future the alerts data are likely to
be made available via a public \api\ under the
\href{\ccByNcSaFourUrl}{Creative Commons
  Attribution-NonCommercial-ShareAlike 4.0} license. Ideally you will
also license the magnetometer data under the
\href{\ccByNcSaFourUrl}{Creative Commons
  Attribution-NonCommercial-ShareAlike 4.0} license, remembering to
define your own attribution requirements.

\helpbox{You may choose a more permissive license, such as without the
  attribution, and\slash or non-commercial clauses, but if it includes
  the share-alike clause you must also dual-license it to AuroraWatch
  UK under the \href{\ccByNcSaFourUrl}{Creative Commons
    Attribution-NonCommercial-ShareAlike 4.0} license. This is because
  the share-alike clause restricts others from imposing additional
  restrictions. The alternative is to grant AuroraWatch UK permission
  to share data derived from your data under the \ccByNcSaFour\
  license. }

The magnetic field data collected by AuroraWatch UK magnetometers will
be made publically available under the
\href{\ccByNcSaFourUrl}{Creative Commons
  Attribution-NonCommercial-ShareAlike 4.0} license, with a short
embargo period (24 to 48 hours). If your magnetometer data is also
licensed as \ccByNcSaFour\ then AuroraWatch UK will share your data in
the same way.

\section{Methods to upload data}
Two methods to upload data to AuroraWatch UK are supported, using
\href{\rsyncUrl}{rsync} through an \ssh\ tunnel, or by \http. Rsync
contains an algorithm to efficiently transfer only the differences
btween the local and remote files and thus is ideal for transferring
the real-time data files. \ssh\ is used to provide a secure connection
method. However, \ssh\ access may not be possible on some networks
(\eg, school networks). For cases when rsync cannot be used a \http\
upload method is available which emulates some of the behaviour of
rsync; whenever possible only the latest additions to a file are
uploaded. The upload method is normally defined in \filename{/etc/awnet.ini}
configuration file, in the \filename{[upload]} section.

\subsection{Rsync uploads}

As user \rootUser\ edit \filename{/etc/awnet.ini}. At the end of the
file add the \filename{[upload]} section if it is missing, it should
appear as
\begin{Code}
[upload]
method = rsync
\end{Code}

As user \piUser\ create the keys for public key authentication:
\begin{Cmd}
ssh-keygen -t dsa
\end{Cmd}
When prompted for the filename to save the key press \myreturn\ to
accept the default. When prompted for the passphrase press \myreturn\
for an empty passphrase. Keep the private key
(\filename{/home/pi/.ssh/id_dsa}) secret, send the public key
(\filename{/home/pi/.ssh/id_dsa.pub}) to AuroraWatch UK.

Create the \ssh\ config file to define hostname and user used for data
transfer. As user \piUser\ edit the file
\filename{/home/pi/.ssh/config}, it should look similar to
\begin{Code}
Host awn-data
Hostname uploadhost
User uploaduser
\end{Code}
You will need to obtain the upload hostname and username from
AuroraWatch UK.

Insert an instruction into the \filename{crontab} file to upload the
data at regular intervals. As user \piUser:
\begin{Cmd}
crontab -e
\end{Cmd}
Add the following lines:
\begin{Code}
### rsync upload
*/3 * * * * nice /home/pi/bin/upload_data.py > /dev/null 2>&1
\end{Code}

\subsection{HTTP uploads}
As user \rootUser\ edit \filename{/etc/awnet.ini}. At the end of the
file add the \filename{[upload]} section if it is missing, it should
appear as
\begin{Code}
[upload]
method = http
url = upload_URL
realm = upload_realm
password = upload_password
\end{Code}
You will need to obtain the upload \URL, realm and password from
AuroraWatch UK.

Insert an instruction the the \filename{crontab} file to upload the
data as regular intervals. As user \piUser:
\begin{Cmd}
crontab -e
\end{Cmd}
Add the following lines:
\begin{Code}[fontsize=\small]
### HTTP upload
# Upload text data for today at regular intervals
*/5 * * * * nice /home/pi/bin/upload_data.py -s today --file-types awnettextdata
# Make several attempts to upload all files from yesterday
5 */6 * * * nice /home/pi/bin/upload_data.py -s yesterday > /dev/null 2>&1
\end{Code}
