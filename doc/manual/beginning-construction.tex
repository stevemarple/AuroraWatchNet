\chapter{Beginning construction}

\section{Anti-static precautions}

\section{Tools required}

\begin{itemize}
\item Soldering iron.
\item Side cutters.
\item Small pliers.
\item In-circuit serial programmer (\isp) for Atmel AVR
  microcontrollers. Instructions are given for the
  \href{http://www.atmel.com/tools/AVRDRAGON.aspx}{Atmel AVR Dragon}
  but other programmers can be used.
\item \usb\ to \ttl\ serial converter for \volt{3.3} operation, \eg, FTDI
  TTL-232R-3V3.
\item Digital multimeter.
\item Solderless breadboard (optional).
\end{itemize}

\section{Order of assembly}

For ease of access components should normally be fitted in order of
increasing size, particularly increasing height. If this order is not
observed it can be very difficult to access the pads of surface mount
devices. It is also preferable that \emph{passive} components
(resistors, capacitors, inductors and crystals) are fitted before
semiconductors (field-effect transistors, integrated circuits). This
is because the semiconductors are easily damaged by electro-static
discharge (sometimes this damage isn't immediately obvious). It is
therefore more convenient to fit as many components as possible before
fitting the semiconductors, at which point \esd\ precautions should be
followed. As field-effect transistors are particularly vulnerable to
damage by \esd\ it is recommended they are fitted as late as possible.
From these guidelines the following order is recommended.
\begin{itemize}
\item Surface-mount passive components.
\item Surface-mount semiconductors.
\item Through-hole passive components.
\item Through-hole semiconductors (\fet s last).
\item Switches.
\item Connectors, battery holders.
\end{itemize}

The first \pcb\ to be assembled is the FLC100 shield, this board
provides the power to the system and will enable you to test each part
correctly.
