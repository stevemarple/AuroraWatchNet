\section{Add a Private Network}

When a separate microcontroller is used for data acquisition it must
be programmed with the IP address or hostname of the Raspberry Pi data
logger. This can be a problem when the full network details are not
known before deployment. In such cases it is helpful to install a
private network between the microcontroller and Raspberry Pi.

These instructions assume that the Raspberry Pi will connect to the
main network using a wired Ethernet connection, and thus a second
Ethernet interface must be fitted. If the main connection is via Wi-Fi
then the original wired Ethernet connection can be used for the
private network and the instructions adapted accordingly.

The second network interface could be provided by a USB Ethernet
adapter. If using the Raspberry Pi 5 then the fastest and most
reliable method is likely a PCI Express network interface card. The
second network interface should be configured with a static IP
address; it is recommended to use a class C network from the 256
contiguous class C networks in the 192.168.0.0 address space. This
example will use the 192.168.200.0/24 private network.

\subsection{Network interface configuration}

\subsubsection{Enable predictable network interface names}

Run
\begin{Cmd}
  sudo raspi-config
\end{Cmd}
from the \code{Advanced Options} menu select the \code{Network
  Interface Names} option to enable predictable network interface
names. Save, exit \code{raspi-config}, and reboot.

Identify the name of the second network interface. The available
interfaces can be found with the command

\begin{Cmd}
  ls /sys/class/net
\end{Cmd}

The built-in Ethernet address is probably \code{end0}, ignore the
loopback interface (\code{lo}) and any wireless interfaces
(\code{wlan*}). This example uses a PCI Express adapter where the
predictable interface name is \code{enP1p1s0}.

\subsubsection{Configure a static IP address}

Configure the second Ethernet interface with a static IP address; this
example will use \code{192.168.200.1}. When using Debian 13 the static
IP address can be configured with \code{netplan} as follows. Create
the file \filename{/etc/netplan/20-enP1p1s0.yaml} (as \rootUser) with
your favourite editor. The contents should be:
\begin{Code}
network:
  version: 2
  renderer: networkd
  ethernets:
    enP1p1s0:
      dhcp4: false
      dhcp6: false
      addresses:
        - 192.168.200.1/24
\end{Code}

Check the file:
\begin{Cmd}
  yamllint /etc/netplan/20-enP1p1s0.yaml
\end{Cmd}

Ignore any errors about the document not starting with \code{---} but
do fix any indentation problems.

\subsection{Install \code{dnsmasq}}

Install the \code{dnsmasq} and \code{dnsmasq-utils} packages to obtain
a local DHCP and caching DNS server:
\begin{Cmd}
  sudo apt update
  sudo apt install dnsmasq dnsmasq-utils
\end{Cmd}

Configure \code{dnsmasq} by adding files (as user \rootUser) into the
directory \filename{/etc/dnsmasq.d}, do not modify the existing
\filename{/etc/dnsmasq.conf} file.

Create the main configuration file,
\filename{/etc/dnsmasq.d/00-main-config.conf}:
\begin{Code}
# Enable dbus for better openresolv support (no need to restart daemon)
# enable-dbus

# Never forward plain names (without a dot or domain part)
domain-needed

# Never forward addresses in the non-routed address spaces.
# bogus-priv

# If you want dnsmasq to listen for DHCP and DNS requests only on
# specified interfaces (and the loopback) give the name of the
# interface (eg eth0) here.
# Repeat the line for more than one interface.
interface=eth1
interface=enP1p1s0

# Set this (and domain: see below) if you want to have a domain
# automatically added to simple names in a hosts-file.
expand-hosts

# Set the domain for dnsmasq. this is optional, but if it is set, it
# does the following things.
# 1) Allows DHCP hosts to have fully qualified domain names, as long
#     as the domain part matches this setting.
# 2) Sets the "domain" DHCP option thereby potentially setting the
#    domain of all systems configured by DHCP
# 3) Provides the domain part for "expand-hosts"
domain=lan
# Note that .lan is a reserved top-level domain name (see
# https://www.ietf.org/archive/id/draft-chapin-rfc2606bis-00.html#rfc.section.2).
# We
# use it here as the config is restricted to the internal LAN.
\end{Code}

Leases are defined in a separate configuration file, 
\filename{/etc/dnsmasq.d/10-leases.conf}:
\begin{Code}
# This file should contain only the fixed leases for dnsmasq

# The DHCP server needs somewhere on disk to keep its lease database.
# This defaults to a sane location, but if you want to change it, use
# the line below.
dhcp-leasefile=/var/lib/misc/dnsmasq.leases

# Uncomment this to enable the integrated DHCP server, you need
# to supply the range of addresses available for lease and optionally
# a lease time. If you have more than one network, you will need to
# repeat this for each network on which you want to supply DHCP
# service.
dhcp-range=192.168.200.100,192.168.200.254,12h
\end{Code}

If the data logger is configured to find the Raspberry Pi server by
hostname (\eg, \code{raspberrypi.example.com}) then it will be
necessary to ensure that \code{dnsmasq} returns the correct IP address
(\code{192.168.2001.}). This can be achieved inserting an entry into
\filename{/etc/hosts}:

\begin{Code}
# Intercept the DNS request for riometer data lgger and have 
# dnsmasq provide our IP address on the internal network.
192.168.200.1   raspberrypi.example.com
\end{Code}

