\chapter[EEPROM settings]{\eeprom\ settings}
\label{chapter:eeeprom-settings}

\newcommand{\optionexample}[1]{\examplebox{\code{#1}}}

\section{Introduction}
The \eeprom\ settings determine the default behaviour of the remote
sensor unit, controlling various features such as the sampling
interval and site ID. The settings are also used to select important
features such as the radio or Ethernet controller. The settings are
described using the Python option name that is passed to the
\filename{generate_eeprom_settings.py} program. Boolean values should
be passed as a \code{0} (false) or \code{1} (true). Numerical values
are interpreted as decimal unless preceded with \code{0x} in which
case they are interpreted as hexadecimal. Settings can be updated with
the \code{send_cmd.py} program; you will need to prefix the names
below with \code{--read-eeprom-} to read the setting, or
\code{--eeprom-} to update the setting. Note that \code{send_cmd.py}
can only update one setting at once, so be sure it has completed
before sending the next command. If changing multiple network settings
then be careful not to reboot the system and end up with a non-working
configuration.

\helpbox{%
  The firmware should indicate on startup what communication methods
  and features are compiled in.

\item The \eeprom\ settings are common to all systems derived from the
  original AuroraWatchNet magnetometer design. This includes the
  prototype cloud detector and the riometer. Hardware support for
  features listed below may not exist for a given hardware design,
  and\slash or the software support may not be compiled into the
  relevant firmware.
}

\warningbox{%
  Misconfiguration of the \eeprom\ settings can prevent the remote
  sensor unit operating correctly, possibly even preventing it from
  communicating with the base unit. If communication cannot be
  reestablished correct settings must be uploaded manually. See the
  section on recovering bad \eeprom\ settings,
  \secname~\ref{sec:bad-eeprom-settings}.
}

\helpbox{Typically the manual you are reading has been tailored for a
  specific instrument type. As there is a great deal of flexibility of
  over which options can be included this section describes all of the
  available options.}

%%% Generic settings %%%
\section{Generic settings}

The settings described in this section are common to all sensor types.

\subsection[adc-ref-type]{\code{adc-ref-type NUMBER}}
\label{sec:eeprom-adc-ref-type}

Select the \adc\ reference that will be used for any house-keeping
data that is sampled by the ATmega1284P's interval \adc. See also
\ref{sec:eeprom-adc-ref-voltage-mv} and
\ref{sec:eeprom-mcu-voltage-mv}.

\subsection[adc-ref-voltage-mv]{\code{adc-ref-voltage-mv ADC_REF_VOLTAGE_MV}}
\label{sec:eeprom-adc-ref-voltage-mv}
Indicate the true voltage of the \adc\ reference used by the
ATmega1284P's internal \adc. Enables correction of internal voltage
references. See \ref{sec:eeprom-adc-ref-type}.

\subsection[aggregate]{\code{aggregate NUMBER}}
\label{sec:eeprom-aggregate}
Select the method used to compute the sample value when oversampling
is in operation. Valid values are:
\begin{enumerate}
\item[0] Mean.
\item[1] Median.
\item[2] Trimmed mean.
\end{enumerate}

\subsection[all-samples]{\code{all-samples BOOLEAN}}
\label{sec:eeprom-all-samples}
When oversampling is in operation report all of the data samples, not
just the aggregated value. Note that this can cause the messages to
become too large to send correctly when a large number of samples are
taken.

\subsection[comms-type]{\code{comms-type NUMBER}}
The installed firmware may support multiple methods of communication
between the remote sensor unit and the base station. If so this option
indicates which method should be used. It is vital that this setting
is correct otherwise the remote sensor unit will not function until
reprogrammed. Valid values are:
\begin{enumerate}
\item[0] XRF radio. Requires \code {XRF} communications option.
\item[1] RFM12B radio. Requires \code {RFM12B} communications option.
\item[2] Ethernet (W5100 or W5500), using \udp\ packets.
\end{enumerate}

\warningbox{Note that automatic selection (255) may not work correctly in all cases and
  is included as an option of last resort (e.g. when encountering an
  unprogrammed \eeprom).
\item For Ethernet support it is necessary to use the correct W5100 or W5500
  firmware.}

\subsection[console-baud-rate]{\code{console-baud-rate BAUDRATE}}
\label{sec:eeprom-console-baud-rate}

The baud rate to be used for debug serial console from the
ATmega1284P.

\warningbox{Higher rates mean that the microcontroller wastes less
  time waiting for serial output to be sent from its single character
  buffer, but requires lower-latency to correctly read incoming
  data. Change with caution and confirm correct operation.}

\subsection[data-quality-input-pin]{\code{data-quality-input-pin PIN}}
\label{sec:eeprom-data-quality-input-pin}
The Arduino pin number that is sensed to detect when data quality
violations occur. With this pin it is possible to use an external
circuit or switch to temporarily pause normal data recording, for
instance so that real time alerts are not generated by activities that
might be expected to cause interference in the data. See also \ref
{sec:eeprom-data-quality-input-active-low}.

\subsection[data-quality-input-active-low]{\code{data-quality-input-active-low PIN}}
\label{sec:eeprom-data-quality-input-active-low}
Indicate if the data quality input pin (see
\ref{sec:eeprom-data-quality-input-pin}) should be regarded as active
during a low logic level.

\subsection[hmac-key]{\code{hmac-key KEY}}
\label{sec:eeprom-hmac-key}

The \hmac-\mdfive\ key used to sign communication messages sent between the
sensor unit and data recording daemon. \textbf{This key should be kept secret.}
The key is a sequence of 16 numbers between 0 and 255 (inclusive). Values may be
given in hexadecimal format if preceded with \code{0x}, otherwise decimal is
assumed. The key used here must match that described in
\secname~\ref{sec:config-file-hmac-key}.

\subsection[magic]{\code{magic STRING}}

String to identify \eeprom\ contents and version. The only recognised value is
\code{AuroraWatch v1.0}.

\subsection[max-message-no-ack]{\code{max-message-no-ack NUMBER}}
\label{sec:eeprom-max-message-no-ack}
The maximum number unacknowledged messages that the sensor unit is
willing to accept. On reaching this number the sensor unit will reboot
in the hope of recovering communications with the data logger. The
count is reset whenever a message acknowledgement is received.See also
\ref{sec:eeprom-max-time-no-ack}.

\subsection[max-messages-led]{\code{max-messages-led NUMBER}}

After startup following some kind of human interaction\footnote{This
  means a manual reset, firmware programming or power
  applied. Watchdog resets and brownout resets are excluded.} the
sensor unit will light its activity \led, and turn it off when the
message acknowledgement has been received, for NUMBER
operations. This is to show the entire system is functioning. A value
of \code{0} indicates that the \led\ should always
operate. \helpbox{On battery-powered sensor units NUMBER should be a
  small non-zero value to limit unnecessary power usage.}

\subsection[max-time-no-ack]{\code{max-time-no-ack SECONDS}}
\label{sec:eeprom-max-time-no-ack}
The maximum duration, in seconds, that the sensor unit is willing run
without receiving an acknowledgement from the base unit. On reaching
this duration the sensor unit will reboot in the hope of recovering
communications with the data logger. The count is reset whenever a
message acknowledgement is received. See also
\ref{sec:eeprom-max-message-no-ack}.


\subsection[mcp7941x-cal]{\code{mcp7941x-cal NUMBER}}

When the internal real time clock is a Microchip MCP7941x set its internal
calibration value to the signed NUMBER. Has no effect for other real time
clocks, or when a software clock is used. If unsure set to \code{0}.

\subsection[mcu-voltage-mv]{\code{mcu-voltage-mv NUMBER}}
\label{sec:eeprom-mcu-voltage-mv}

The operating voltage of the microcontroller unit in millivolts. See
\ref{sec:eeprom-adc-ref-type}.

\subsection[num-samples]{\code{num-samples NUMBER}}
\label{sec:eeprom-num-samples}

Enable oversampling; take NUMBER samples and compute an aggregate
value. See also sections \ref{sec:eeprom-aggregate} and
\ref{sec:eeprom-all-samples}.


\subsection[rtc-device-address]{\code{rtc-device-address ADDRESS}}

The 7 bit \itwoc\ address of the on-board real time clock. The correct
address can usually be sensed automatically. Set to 255 (\code{0xFF})
to disable and use automatic selection.

\subsection[rtcx-device-type]{\code{rtcx-device-type NUMBER}}
Override the automatic real time clock device identification. This may
be required for non-standard \itwoc\ addresses or devices. Valid
values are:
\begin{enumerate}
\item[0] Dallas Semiconductor DS1307 or compatible devices (such as
  DS1338).
\item[1] Devices in the Microchip MCP7941x family of real time clocks.
\item[2] Philips PCF85263 real time clock.
\item[255] Enable automatic device type selection.
\end{enumerate}



\subsection[sampling-interval-16th-s]{\code{sampling-interval-16th-s DURATION}}
\label{sec:eeprom-sampling-interval}

The initial sampling interval, defined in units of
$\frac{1}{16}$\SI{}{\second}. The \filename{send_cmd.py} command, when
used with its \code{--sampling-interval} option, can be used to modify
sampling interval without requiring a reboot.

\subsection[site-id]{\code{site-id NUMBER}}
Set the unique site identifier number, in the range 0 to 65535 inclusive.

\subsection[vin-divider]{\code{vin-divider NUMBER}}

An integer value indicating the scaling reduction applied to the input voltage
before measurement. That is, the actual measured value should be multiple by
NUMBER to obtain the true input voltage.


\section{XRF communication settings}

The settings described in this section are common to all systems
which use (or can use) a Ciseco XRF radio module. The firmware must be
compiled with \code{COMMS_XRF} communications support enabled.

\subsection[radio-local-id]{\code{radio-local-id NUMBER}}

Local radio identifier for XRF radio communication.

\subsection[radio-remote-id]{\code{radio-remote-id NUMBER}}

Remote radio identifier for XRF radio communication. 

\subsection[radio-xrf-band]{\code{radio-xrf-band NUMBER}}

Select the XRF radio band (1 to 7 inclusive). Not implemented in
existing firmware.

\subsection[radio-xrf-channel]{\code{radio-xrf-channel NUMBER}}

Select the XRF radio channel. 


\section{RFM12B communication settings}

The settings described in this section are common to all systems
which use (or can use) a Hope RFM12B module. The firmware must be
compiled with \code{COMMS_RFM12B} communication support enabled.

\subsection[radio-rfm12b-band]{\code{radio-rfm12b-band NUMBER}}

Select the band number to use for radio communications. 

\subsection[radio-rfm12b-channel]{\code{radio-rfm12b-channel NUMBER}}

Select the channel number to use for radio communications.


\section{Ethernet communication settings}

The settings described in this section are common to all systems
which use (or can use) Ethernet for communication to the base unit.
The firmware must be compiled with \code{W5100} or \code{W5500}
communications support enabled.

\subsection[dns1, dns2, dns3]{\code{dns1 A,B,C,D}\newline \code{dns2 A,B,C,D}\newline \code{dns3 A,B,C,D}}
The sensor unit supports the use of up to three \dns\ servers for host
name resolution. Each server \ip\ address must be set separately as a
comma-separated list (\code{A, B, C, D}).

Set the \dns\ server \ip\ addresses used for name resolution, up to
three different servers may be specified. When \dhcp\ configuration is
in operation any \dns\ servers given take priority. If only two
servers are given by \dhcp\ and a third server is specified in the
\eeprom\ settings it will also be used. It is permissible to include
repeated server addresses. To disable a particular \dns\ setting set
to \code{255, 255, 255, 255}.

\subsection[gateway]{\code{gateway A,B,C,D}}

Set the network gateway, given as a comma-separated list of 4
integers. The value is ignored when \dhcp\ is in operation.

\subsection[local-ip-address]{\code{local-ip-address A,B,C,D}}

The local \ip\ address, given as a comma-separated list of 4
integers. Use the value \code{255, 255, 255, 255} to indicate an
address should be obtained using \dhcp.

\subsection[local-ip-port]{\code{local-ip-port NUMBER}}
The \udp\ port number from which the sensor unit will communicate.

\subsection[local-mac-address]{\code{local-mac-address A,B,C,D,E,F}}

Assign the \macAddress\ that the sensor unit should use on the
network, give as a comma-separated list of 6 integers (you may need to use the
\code{0x} prefix on each value to indicate hexadecimal).
\warningbox{\macAddress es must be unique on a network. You
  may use one assigned by the original equipment manufacturer for the
  Ethernet shield or a locally-assigned value. If using a
  locally-assigned \macAddress\ check with your network manager that
  the assignment is unique.}

\subsection[netmask]{\code{netmask A,B,C,D}}

Set the network mask, given as a comma-separated list of 4
integers. The value is ignored when \dhcp\ is in operation.


\subsection[remote-hostname]{\code{remote-hostname HOSTNAME}}
\label{sec:eeprom-remote-hostname}

Set the hostname of the data logger to which the sensor unit should send its
\udp\ data packets. The hostname is looked up once from \dns\ at startup. If
multiple values are returned from \dns\ only the first value is used. If the
name cannot be looked up then a fallback \ip\ address may be used, see
\ref{sec:eeprom-remote-ip-address}. Requires \code {W5100} or \code{W5500}
communication options.


\subsection[remote-ip-address]{\code{remote-ip-address A,B,C,D}}
\label{sec:eeprom-remote-ip-address}

Set the remote \ip\ address to which the sensor unit should send its
\udp\ data packets, given as a comma-separated list of 4 integers. Note that is
a remote hostname is set (\ref{sec:eeprom-remote-hostname}), and can be
resolved, its value takes
precedence. Requires \code {W5100} or \code{W5500}
communication options.


\subsection[remote-ip-port]{\code{remote-ip-port NUMBER}}

Set the remote \ip\ port number to which the sensor unit should send
its \udp\ data packets. Replies are expected to originate from the
same port number.  Requires \code {W5100} or \code{W5500}
communication options.


%%% MCP342x settings %%%
\section{MCP342x ADC settings}
\label{sec:eeprom-mcp342x-adc}

The settings described in this section are used to configure Microchip
MCP342x \adc s. Depending on the hardware there may be one or more
\adc s of this type present. The firmware must be compiled with
code{FEATURE_FLC100} or \code{FEATURE_RIOMETER} features enabled.

\subsection[adc-address-list]{\code{adc-address-list ADDRESS_LIST}}

A comma-separated list of the \itwoc\ address(es) for the sensor
MCP3424 \adc(s).  Addresses may be given in hexadecimal format if
preceded with \code{0x}, otherwise decimal is assumed. Missing \adc s
should be indicated by giving the value 255.

\optionexample{adc-address-list 0x6E,0x6A,0x6C}

\subsection[adc-channel-list]{\code{adc-channel-list CHANNEL_LIST}}

A comma-separated list of the channels for the sensor MCP3424
\adc(s). Unless there is a good reason to swap the \adc s all channels
should be 1.

\optionexample{adc-channel-list 1,1,1}

\subsection[adc-gain-list]{\code{adc-gain-list GAIN_LIST}}

A comma-separated list of the gains for the sensor MCP3424
\adc(s). Valid values for the gain are 1, 2, 4, and 8. If the gain is
set too high there is a possibility that the output value will
saturate. %
\ifIsInstrument{magnetometer}{%
  With the standard \nT{100000} range FLC100 sensors the gain should
  normally be set to 1 or 2. With HEZ alignment the E channel may be
  set as high as 8.}

\optionexample{adc-channel-list 1,8,1}

\subsection[adc-resolution-list]{\code{adc-resolution-list BITS}}

A comma-separated list of the resolutions for the sensor MCP3424
\adc(s). Valid values for the resolution are 12, 14, 16, and
18. Higher resolution sampling requires longer sample times, consult
for the MCP3424 data sheet for more information.

\section{GNSS settings}

When \gnss\ support is included the sensor unit can derive and
maintain accurate time keeping independently of the base unit. The
firmware must be compiled withe the \code{FEATURE_GNSS} feature
enabled.

\subsection[gnss-default-baud-rate]{\code{gnss-default-baud-rate BAUDRATE}}

This setting is the default baud rate of the \gnss\ module immediately
after power is applied. If the desired baud rate for the \gnss\ module
differs then the microcontroller must communicate with the \gnss\
module at this baud rate initially in order to instruct it to change
its baud rate.

\subsection[gnss-desired-baud-rate]{\code{gnss-desired-baud-rate BAUDRATE}}

This setting is the desired baud for the microcontroller and \gnss\
module to communicate during normal operation.

\subsection[use-gnss]{\code{use-gnss BOOLEAN}}

Indicates if \gnss\ timing should be used whenever possible. If set
the microcontroller will still fall back to other timing sources if
the \gnss\ result is not valid.

\section{FLC100 magnetometer settings}

The FLC100 magnetometer settings affect the operation for systems
which use the Stefan Mayer Instruments FLC100 magnetometer
sensor(s). For data sampling settings see
\ref{sec:eeprom-mcp342x-adc}.

\subsection[flc100-power-up-delay-50ms]{\code{flc100-power-up-delay-50ms DELAY}}
Set the delay, in units of \SI{50}{\milli\second}, from powering on
the FLC100 magnetometer sensor to when it is ready for samples to be
taken. Set to \code{0} to disable power control and leave powered
permanently.

\subsection[flc100-present]{\code{flc100-present BOOLEAN}}
Indicate if FLC100 magnetometer sensor(s) are present.

\section{Riometer settings}
The data logger is capable of supporting imaging riometers of up to 64
beams (8 \ensuremath{\times} 8). The number of rows
($N_\mathrm{rows}$) and columns ($N_\mathrm{cols}$) are specified
independently, although normally they would be the same value
(scanning systems where $N_\mathrm{rows} = 1$ or $N_\mathrm{cols} = 1$
are supported) . After the Butler matrix has been commanded to output
a different riometer row the sampling must be delayed in order for the
riometers to settle to the new signal value. During this settling time
the Microchip MCP324x \adc s can be instructed to record additional
house-keeping data such as riometer operating voltages or data logger
temperature. The number of house-keeping `slots' available is the same
as the number of imaging riometer rows. For programming convenience
the rows are numbered from zero. Each slot can be configured
independently, to avoid repetition in the riometer settings $n$ $n$ is
used to indicate the house-keeping slot number where
$0 <= n < N_\mathrm{rows}$, $n \in \mathbb{Z}$.

It follows that for a wide beam riometer system (1 \ensuremath{\times}
1) only a single house-keeping slot is available. Values in unused rows and
columns should be set to 255 (\code{0xFF}), which is the value of
unprogrammed \eeprom\ memory.

Riometer support requires the firmware to be compiled with
\code{FEATURE_RIOMETER} enabled.

For data sampling settings see also \ref{sec:eeprom-mcp342x-adc}.

\subsection[generic-adc-address-list]{\code{generic-adc-address-list ADDRESS_LIST}}
Set the \adc\ \itwoc\ addresses used for each \adc. Order is
important, the first address dictates which\adc\ is considered the
first. Addresses defined here apply for house-keeping data captured
during the riometer house-keeping slots and the main riometer data.

\subsection[rio-present]{\code{rio-present BOOLEAN}}
Indicate if the riometer \adc\ board is present.

\subsection[rio-housekeeping-n-adc-channel-list]{\code{rio-housekeeping-$n$-adc-channel-list CHANNEL_LIST}}
Define the MCP324x \adc\ channel numbers that are to be used for each
\adc\ during house-keeping slot $n$. Channel numbers should be given
as a comma-separated list of the values 1, 2, 3, or 4.

\subsection[rio-housekeeping-n-adc-gain-list]{\code{rio-housekeeping-$n$-adc-gain-list GAIN_LIST}}
Define the MCP324x \adc\ gain values that are to be used for each
\adc\ during house-keeping slot $n$. Gains should be given as a
comma-separated list of the values 1, 2, 4, or 8.

\subsection[rio-housekeeping-n-adc-mask]{\code{rio-housekeeping-$n$-adc-mask ADC_MASK}}
A bit mask defining which \adc s are to be used during house-keeping
slot $n$. Bit 0 (\lsb) is the first \adc, bit 7 the last possible
\adc.

\subsection[rio-housekeeping-n-adc-resolution]{\code{rio-housekeeping-$n$-adc-resolution NUMBER}}
Define the MCP324x \adc\ resolution value that is to be used for all
active \adc s during house-keeping slot $n$. The resolution should be
given as a single number from the values 12, 14, 16, or 18.

\helpbox{As the resolution determines the conversion time, at active
  \adc s will use the same resolution.}

\subsection[rio-housekeeping-n-num-samples]{\code{rio-housekeeping-$n$-num-samples NUMBER}}

Define the number of samples to be taken for the house-keeping data
during slot $n$. NUMBER should be at least 1 and a maximum of 16.


\subsection[rio-housekeeping-n-aggregate]{\code{rio-housekeeping-$n$-aggregate NUMBER}}

Define the method by which multiple samples should be reduced to a
single value. See the description in \ref{sec:eeprom-aggregate}.

\subsection[rio-riometer-adc-channel-list]{\code{rio-riometer-adc-channel-list CHANNEL_LIST}}

Define the MCP324x \adc\ channel numbers that are to be used for each
\adc\ during the main riometer data acquisition. Channel numbers
should be given as a comma-separated list of the values 1, 2, 3, or 4.

\subsection[rio-riometer-adc-gain-list]{\code{rio-riometer-adc-gain-list GAIN_LIST}}

Define the MCP324x \adc\ gain values that are to be used for each
\adc\ during the main riometer data acquisition. Gains should be given
as a comma-separated list of the values 1, 2, 4, or 8.

\subsection[rio-riometer-num-samples]{\code{rio-riometer-num-samples NUMBER}}

Define the number of samples to be taken during the main riometer data
acquisition. NUMBER should be at least 1 and a maximum of 16.

\subsection[rio-riometer-aggregate]{\code{rio-riometer-aggregate NUMBER}}

Define the method by which multiple samples of the main riometer data
should be reduced to a single value. See the description in
\ref{sec:eeprom-aggregate}.

\subsection[rio-riometer-adc-resolution]{\code{rio-riometer-adc-resolution RESOLUTION}}

Define the MCP324x \adc\ resolution value that is to be used for all
active \adc s during the main riometer data acquisition. The
resolution should be given as a single number from the values 12, 14,
16, or 18.

\helpbox{As the resolution determines the conversion time, at active
  \adc s will use the same resolution.}


\subsection[rio-riometer-adc-mask]{\code{rio-riometer-adc-mask ADC_MASK}}

A bit mask defining which \adc s are to be used during the main
riometer data acquisition. Bit 0 (\lsb) is the first \adc, bit 7 the
last possible \adc.

\subsection[rio-num-rows]{\code{rio-num-rows NUMBER}}

Set the number of rows in the imaging riometer. NUMBER must be at
least 1, and no more than 8.

\subsection[rio-num-columns]{\code{rio-num-columns NUMBER}}

Set the number of rows in the imaging riometer. NUMBER must be at
least 1, and no more than 8.

\subsection[rio-scan-pins]{\code{rio-scan-pins PINS}}
\label{sec:eeprom-rio-scan-pins}

Define the Arduino pins used for the riometer scan control. The pin numbers
should be given as a comma-separated list of integers.

\subsection[rio-presample-delay-ms]{\code{rio-presample-delay-ms DELAY_MS}}

Define the delay, in milliseconds, from when the imaging riometer row has been
selected to when riometer sampling starts. For widebeam or scanning riometers
without row scanning the delay should be set to 0.

\subsection[rio-row-scan-interval-ms]{\code{rio-row-scan-interval-ms INTERVAL_MS}}

Define the interval, in milliseconds, between riometer row scans. For widebeam
or scanning riometers without row scanning the delay should be set to 0.

\subsection[rio-gpio-address]{\code{rio-gpio-address ADDRESS}}

Define the 7 bit \itwoc\ address of the Microchip MCP23008 IO expander. Only
applicable to imaging riometer systems using the external 8 input \adc\ board.
Set to 255 (\code{0xFF}) for other systems.

\subsection[rio-scan-mapping]{\code{rio-scan-mapping MAPPING}}
\label{sec:eeprom-rio-scan-mapping}

Define the mapping from riometer row number to the pin outputs required to
select the row. The mapping must be given as a comma-separated list of integers.
See also \ref{sec:eeprom-rio-scan-pins}.
\section{Honeywell HIH61xx humidity sensor}

A Honeywell HIH61xx humidity sensor may be present in the cloud
detector or imaging riometer systems. Support requires
\code{FEATURE_HIH61XX_WIRE} to be enabled for hardware \itwoc\ support
or \code{FEATURE_HIH61XX_SOFTWIRE} to be enabled for software \itwoc\
support.

\subsection[hih61xx-present]{\code{hih61xx-present BOOLEAN}}

Indicate if a Honeywell HIH61xx humidity sensor is fitted. Requires

\ifIsInstrument{riometer}{Hardware support is present only for imaging
  riometer systems. Firmware support may be present for all riometer
  systems.}

\section{Melexis MLX90614 settings}

The Melexis MLX90614 digital non-contact thermometer is used by the
cloud detector for measurements of sky temperature. Support requires
\code{FEATURE_MLX90614} feature to be enabled.


\subsection[mlx90614-present]{\code{mlx90614-present BOOLEAN}}

Indicate if the Melexis MLX90614 digital non-contact thermometer is fitted.

\section{Lightning sensor settings}

Lightning detection using the Austria Micro Systems AMS3935 sensor
requires the firmware to be compiled with \code{FEATURE_AS3935}
enabled.

\subsection[as3935-present]{\code{as3935-present BOOLEAN}}
Indicate if Austria Micro Systems AS3935 lightning detector is
fitted. Hardware support is not available for all designs. Requires
\code{FEATURE_AS3935} for firmware support.


\section{Environmental settings}
Some systems may include options to help regulate the internal sensor
unit temperature.

\subsection[fan-pin]{\code{fan-pin NUMBER}}
The Arduino pin number that should be switched when cooling fan
support is included. Set to 255 (\code{0xFF}) to
disable this option.

\subsection[fan-temperature]{\code{fan-temperature NUMBER}}
The temperature (in \degC{}) at which the cooling fan should be
enabled. When multiple temperature sensors are included it is the
house-keeping system temperature value that is used.

\subsection[fan-temperature-hysteresis]{\code{fan-temperature-hysteresis NUMBER}}
The hysteresis (in \degC{}) used when cooling fan support is included.

\subsection[heater-pin]{\code{heater-pin PIN}}
The Arduino pin used to enable the heater. Set to 255 (\code{0xFF}) to
disable this option.

\section[(Micro)SD card settings]{(Micro)\sd\ card settings}

An experimental feature exists to save data to a \sd\ card instead of
sending to the base unit. This was intended for use in magnetometer
site surveys and is not optimized for production usage or low-power
operation. Requires \code{FEATURE_SD_CARD} feature to be enabled.

\warningbox{Some hardware designs contain a \sd\ card interface that
  can only be used when the operating voltage is set to \volt{3.3}. Do
  not attempt to use such an interface when the operating voltage is
  set to \volt{5}. Ethernet operation generally requires \volt{5}
  operation and so is incompatible with this feature.}


\subsection[sd-select]{\code{sd-select PIN}}

The Arduino pin number connected to the select line of the \sd\ card interface.
Requires \code{FEATURE_SD_CARD}.

\subsection[use-sd]{\code{use-sd BOOLEAN}}

Indicate if the \sd\ card interface (if fitted) should be used. Requires
\code{FEATURE_SD_CARD} for firmware support.



%%% ======================================
\section[Recovering bad EEPROM settings]{%
  Recovering bad \eeprom\ settings}
\label{sec:bad-eeprom-settings}

The easiest method to recover from bad \eeprom\ settings is to generate new
settings using the \filename{generate_eeprom_image.py} program. The original
settings used at manufacture should only be used if the firmware has not been
updated, otherwise the settings may be missing values required by the newer
firmware. Further information on generating new \eeprom\ settings can be found
by running the \code{generate_eeprom_image.py} program with the \code{--help}
option.

