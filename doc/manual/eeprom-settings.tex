\chapter[EEPROM settings]{\eeprom\ settings}

\newcommand{\optionexample}[1]{\examplebox{\code{#1}}}

\section{Introduction}
The \eeprom\ settings determine the default behaviour of the remote
sensor unit, controlling various features such as the sampling
interval and site ID. The settings are also select important
features such as the radio or ethernet controller.

\warningbox{%
  Misconfiguration of the \eeprom\ settings can prevent the remote
  sensor unit operating correctly, possibly even preventing it from
  communicating with the base unit. If communication cannot be
  restablished correct settings must be uploaded manually. See the
  section on recovering bad \eeprom\ settings,
  \secname~\ref{sec:bad-eeprom-settings}.
}


\section{Settings}
The settings are described below using the Python option name that is
passed to the \filename{generate_eeprom_settings.py} and
\filename{send_cmd.py} programs.

\subsection{\code{--eeprom-adc-address-list}}
A comma-separated list of the 3 \itwoc\ addresses for the magnetometer
sensor \adc(s). Addresses may be given in hexadecimal format if
preceded with \code{0x}, otherwise decimal is assumed. Missing \adc s
should be indicated by giving the value 255. 

\optionexample{--eeprom-adc-address-list 0x6E,0x6A,0x6C}

\subsection{\code{--eeprom-adc-channel-list}}
A comma-separated list of the 3 \itwoc\ channels for the magnetometer
sensor \adc(s). Unless there is a good reason to swap the \adc s all
channels should be 1.

\optionexample{--eeprom-adc-channel-list 1,1,1}

\subsection{\code{--eeprom-adc-gain-list}}
A comma-separated list of the 3 \itwoc\ gains for the magnetometer
sensor \adc(s). Valid values for the gain are 1, 2, 4, and 8. If the
gain is set too high there is a possibility that the output value will
saturate. With the standard \nT{100000} range FLC100 sensors the gain
should normally be set to 1 or 2. With HEZ alignment the E channel may
be set as high as 8.

\optionexample{--eeprom-adc-channel-list 1,8,1}

% \subsection{\code{--eeprom-adc-ref-type EEPROM_ADC_REF_TYPE
% \subsection{\code{--eeprom-adc-ref-voltage-mv EEPROM_ADC_REF_VOLTAGE_MV
% \subsection{\code{--eeprom-adc-resolution-list BITS
\subsection{\code{--eeprom-aggregate {0,1,2}}}
\label{sec:eeprom-aggregate}
Select the method used to compute the sample value when oversampling
is in operation. Valid values are:
\begin{enumerate}
\item[0] Mean.
\item[1] Median.
\item[2] Trimmed mean.
\end{enumerate}
\subsection{\code{--eeprom-all-samples {0,1}}}
\label{sec:eeprom-all-samples}
When oversampling is in operation have all data samples returned. Note
that this can cause the messages to become too large to send correctly
when a large number of oversamples are taken, particularly with a 3
axis magnetometer system.
% \subsection{\code{--eeprom-as3935-present {0,1}

\subsection{\code{--eeprom-comms-type {0,1,2}}}
The installed firmware may support multiple methods of communication
between the remote sensor unit and the base station. If so this option
indicates which method should be used. It is vital that this setting
is correct otherwise the remote sensor unit will not function until
reprogrammed. Valid values are:
\begin{enumerate}
\item[0] XRF radio.
\item[1] RFM12B radio.
\item[2] W5100 Ethernet, using \udp\ packets.
\end{enumerate}
Note that automatic selection may not work correctly in all cases and
is included as an option of last resort (e.g. when encountering an
unprogrammed \eeprom).

% \subsection{\code{--eeprom-fan-pin EEPROM_FAN_PIN
% \subsection{\code{--eeprom-fan-temperature EEPROM_FAN_TEMPERATURE
% \subsection{\code{--eeprom-fan-temperature-hysteresis EEPROM_FAN_TEMPERATURE_HYSTERESIS
% \subsection{\code{--eeprom-flc100-power-up-delay-50ms DURATION_50ms
% \subsection{\code{--eeprom-flc100-present {0,1}
% \subsection{\code{--eeprom-hih61xx-present {0,1}
\subsection{\code{--eeprom-hmac-key EEPROM_HMAC_KEY}}
\label{sec:eeprom-hmac-key}
The \hmac-\mdfive\ key used to sign communication messages sent
between the magnetometer and data recording daemon. \textbf{This key
  should be kept secret.} The key is a sequence of 16 numbers between
0 and 255 (inclusive). Values may be given in hexadecimal format if
preceded with \code{0x}, otherwise decimal is assumed. The key used
here must match that described in
\secname~\ref{sec:config-file-mag-key}.



% \subsection{\code{--eeprom-local-ip-address IP_ADDRESS
% \subsection{\code{--eeprom-local-ip-port NUMBER
% \subsection{\code{--eeprom-local-mac-address MAC_ADDRESS
% \subsection{\code{--eeprom-magic STRING
% \subsection{\code{--eeprom-max-message-no-ack NUMBER
% \subsection{\code{--eeprom-max-messages-led NUMBER
% \subsection{\code{--eeprom-mcp7941x-cal NUMBER
% \subsection{\code{--eeprom-mcu-voltage-mv NUMBER
% \subsection{\code{--eeprom-mlx90614-present {0,1}
\subsection{\code{--eeprom-num-samples NUMBER}}
\label{sec:eeprom-num-samples}
Enable oversampling; take NUMBER samples and compute an aggregate
value. See also sections {sec:eeprom-aggregate} and
{sec:eeprom-all-samples}.
% \subsection{\code{--eeprom-radio-local-id LOCAL_ID
% \subsection{\code{--eeprom-radio-remote-id REMOTE_ID
% \subsection{\code{--eeprom-radio-rfm12b-band BAND_NUMBER
% \subsection{\code{--eeprom-radio-rfm12b-channel CHANNEL_NUMBER
% \subsection{\code{--eeprom-radio-xrf-band {1,2,3,4,5,6}
% \subsection{\code{--eeprom-radio-xrf-channel CHANNEL_NUMBER
% \subsection{\code{--eeprom-remote-ip-address IP_ADDRESS
% \subsection{\code{--eeprom-remote-ip-port NUMBER
\subsection{\code{--eeprom-sampling-interval-16th-s DURATION}}
\label{sec:eeprom-sampling-interval}
The initial sampling interval, defined in units of
$\frac{1}{16}$\SI{}{s}. The \filename{send_cmd.py} command, when used with
its \code{--sampling-interval} option, can be used to modify sampling
interval without requiring a reboot.

% \subsection{\code{--eeprom-sd-select EEPROM_SD_SELECT
\subsection{\code{--eeprom-site-id EEPROM_SITE_ID}}
Set the unique site identifier number.
% \subsection{\code{--eeprom-use-sd {0,1}
% \subsection{\code{--eeprom-vin-divider NUMBER

\section[Recovering bad EEPROM settings]{%
  Recovering bad \eeprom\ settings}
\label{sec:bad-eeprom-settings}

The easiest method to recover from bad \eeprom\ settings is to
generate new settings using the \filename{generate_eeprom_image.py}
program. The original settings used at manufacture should only be used
if the firmware has not been updated, otherwise the settings may be
missing values required by the newer firmware. Further information on
generating and uploading new \eeprom\ settings can be found in
\secname~\ref{sec:generate-eeprom-settings}.

