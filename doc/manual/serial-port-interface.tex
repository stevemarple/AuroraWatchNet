\chapter{Serial port interface}

\section{Introduction}

The sensor unit microcontroller contains a serial port interface that
can be used for observing its network status and debugging its
behaviour. By default only a limited set of information is printed,
since the time taken to print could interfere with the data
acquisition tasks of the microcontroller.

The serial port also accepts commands that can be used to modify the
system behaviour. Access to the serial port is not required for normal
operation, and \eeprom\ settings
(\chaptername~\ref{chapter:eeeprom-settings}) should normally be
changed using the \code{send_cmd.py} command.

\section{Connection information}

The serial port interfaces sends 8 bit data, with no parity and 1 stop
bit (8N1). The normal baud rate is 115200 baud, or 9600 baud for
battery-powered systems. The default baud rate can be changed with an
\eeprom\ setting (\ref{sec:eeprom-console-baud-rate}).

\section{Commands}

The serial port accepts the following commands.

\helpbox{%
  Numerical values are interpreted as hexadecimal when preceded by
  \code{0x} or \code{0X}, and interpreted as octal when preceded by
  \code{0}. Number which do no start with \code{0x}, \code{0X} or
  \code{0} are interpreted as decimal.
}

\subsection[eepromRead]{\code{eepromRead=ADDRESS,NUMBER}}

Print the value of NUMBER bytes of \eeprom\ memory, starting at
address ADDRESS.

\subsection[eepromRead]{\code{eepromWrite=ADDRESS,VALUES}}

Set the value of \eeprom\ memory, starting at address ADDRESS. If the
VALUES string is prefixed with \code{'} then the remaining values are
interpreted as a literal string. Otherwise the VALUES string is
interpreted as a comma-separated list of integer values, each of which
must be in the range 0 to 255 inclusive. Values can be independently
prefixed as described above and will then be interpreted as octal,
decimal, or hexadecimal as appropriate. After writing to \eeprom\
memory the values are read back and printed.

\warningbox{%
  This command is analogous to the early BASIC command \code{POKE},
  but more dangerous since its effects last beyond a reboot. Use with
  extreme caution! The Raspberry Pi \code{send_cmd.py} command
  performs the same role but with a safer interface and should be
  preferred over \code{eepromWrite}.
}


\subsection[verbosity]{\code{verbosity}\newline\code{verbosity=NUMBER}}

The bare \code{verbosity} command can be used to print the current
verbosity level. The assignment form is used to set the verbosity
level to NUMBER. Valid values are:
\begin{enumerate}
\item[0] Minimal console output. Sensor data values are not
  printed. \dhcp\ messages about renewal, rebinding and failures are
  always printed (where applicable).
\item[1] Print basic sensor and house-keeping data values. This is the
  default setting.
\item[>1] Print the local clock and server clock times.
\item[10] Print the server acknowledgement packet.
\item[11] Print the raw data values used to obtain the aggregate
  data value.
\item[12] Print \nmea\ sentences from the \gnss\ receiver. The
  firmware must be compiled withe the \code{FEATURE_GNSS} feature
  enabled.
\end{enumerate}

\subsection[reboot]{\code{reboot=TRUE}}

Reboot the microcontroller after first printing the message
\code{Reboot command received from console}. The command must be given
exactly as shown, with \code{TRUE} in upper-case.

\subsection[samplingInterval_16th_s]{\code{samplingInterval_16th_s}\newline%
  \code{samplingInterval_16th_s=NUMBER}}

Print or set the data sampling interval in units of
$\frac{1}{16}$\SI{}{\second}. The data sampling change will occur
immediately on the next sampling period and unlike the comparable
\eeprom\ setting (\ref{sec:eeprom-sampling-interval}) does not require
a microcontroller reboot. Not applicable to riometer systems.

\helpbox{%
  This setting does not persist after a reboot. Use the \eeprom\
  setting (\ref{sec:eeprom-sampling-interval}) to make the sampling
  interval change permanent.  }

\subsection[useSd]{\code{useSd}\newline%
  \code{useSd=NUMBER}}

Query or enable\slash disable saving data to the \sd\ card. When
NUMBER is \code{0} saving data to the \sd\ card is disabled, and any other
value enables. The firmware must be compiled with the
\code{FEATURE_SD_CARD} feature enabled.
