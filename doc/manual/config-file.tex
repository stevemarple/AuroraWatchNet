\chapter{Configuration file options}

% \newcommand{\configexample}[1]{\examplebox{\code{#1}}}
\newcommand{\configexample}[1]{\examplebox{#1}}

\section{Introduction}
Many of the AuroraWatchNet programs read a common configuration file,
typically located at \filename{/etc/awnet.ini}. The configuration file
is broken into sections, each of which starts with a section header in
square brackets (\code({[like_this]}). Other line contain key names
and values, written like \code{key = value}. Leading and trailing
whitespace around both the key and value is ignored. Section headers
and key names do not contain whitspace, words may be separated with an
underscore). The configuration file can also contain comments, which
are entered with a hash (\code{\#}) or semi-colon (\code{;}) as the
first character.

The configuration file is parsed using Python's SafeConfigParser
module. This allows values defined in the same section, or in the
\code{[DEFAULT]} section to be inserted into other key value
definitions.

\section{\code{[DEFAULT]}}

The \code{[DEFAULT]} section is special as it defines values which can
used elsewhere in the configuration file.

\subsection{site}
Define the site code, typically a three letter abbreviation. This is
used elsewhere within the configuration file, \eg, data filenames.

Default: none.

\configexample{\code{site = lan1}}

\section{\code{[awnettextdata]}}

Options associated with the standard text-format output data file.

\subsection{filename}
\label{sec:config-data-filename}
Define the filename used for text-format data files. This string is
expanded as a \code{strftime} format string and accepts the normal
\code{strftime} format specifiers. \todo{list strftime format
  specifiers somewhere}. However, since Python expands the string
first any percent characters used as part of a \code{strftime} format
specifier must be repeated.

Default: none.\\
Typically set to \code{filename = /data/aurorawatchnet/\%(site)s/\%\%Y/\%\%m/\%(site)s_\%\%Y\%\%m\%\%d.txt}

\configexample{Define the filename, \code{\%(site)} is replaced with
  the site abbreviation which was defined previously in the
  \code{[DEFAULT]} section. Notice how the \code{strftime} format
  specifiers require two \% characters.\newline 
  \code{filename = /data/aurorawatchnet/\%(site)s/\%\%Y/\%\%m/\%(site)s_\%\%Y\%\%m\%\%d.txt}\newline%
  For June 20th 2014 and the site \code{cwx} this would expand to\newline %
  \code{filename = /data/aurorawatchnet/cwx/2014/06/cwx_20140620.txt}
}

\section{\code{[awpacket]}}
Options associated with the standard binary output format. This format
is incovenient to read but preserves the received data messages from
the magnetometer, and the responses sent back from the recording
daemon. It is possible to play back these files to the recording
daemon and regenerate other data formats.

\subsection{filename}

See section~\ref{sec:config-data-filename} for a description.

Default: none\\
Typically set to \code{filename = /data/aurorawatchnet/\%(site)s/\%\%Y/\%\%m/\%(site)s_\%\%Y\%\%m\%\%d.awp}

\subsection{key}
By default the binary data packets are written out with their original
signing key. If you plan to make the binary data format available you
should then for security reasons you should probably set a different
signing key. You will then be able to share this key without
compromising the communication channel with the magnetometer. The key
is a 32 character hexadecial string, without any \code{0x} prefix.

Default: none
Typically set to a simple code, \code{key = 00000000000000000000000000000000}

\section{\code{[logfile]}}
Options associated with the recorded log files. 

\subsection{filename}

See section~\ref{sec:config-data-filename} for a description.

Default: none\\
Typically set to \code{filename = /data/aurorawatchnet/\%(site)s/\%\%Y/\%\%m/\%(site)s_\%\%Y\%\%m\%\%d.log}


\section{\code[daemon]}
Options relating to the data-recording daemon.

\subsection{\code{connection}}
Defines the connection between the daemon and the magnetometer. Radio
communication emulates a serial port connection. If the magnetometer
is \PoE\ model the communication should be set to \code{ethernet}.

Default: \code{serial}

\section{\code{[serial]}}
Options relating to serial data communication between the magnetometer
and data recording daemon. This section is not used when communication
uses ethernet.

\subsection{\code{port}}
The filename of the serial port. For the Ciseco \emph{Slice of Radio} module
this should be set to \filename{/dev/ttyAMA0}, for the Ciseco
\emph{URF} module this will probably be \filename{/dev/ttyACM0}.

Default: \filename{/dev/ttyACM0}

\subsection{\code{baudrate}}
Baud rate used with the serial port. For the Ciseco \emph{Slice of
  Radio} module this should be set to \code{9600}, for the Ciseco
\emph{URF} module a higher rate of \code{57600} can be used.

Default: \code{9600}

\subsection{\code{setup}}
The set-up string which should be sent to the serial device, for
defining the channel number used for communication \etc.

Default: none.\\
Typically set to \code{ATRE;ATCN 25;ATLI R;ATAC}. Ensure that the
channel number defined here matches the setting programmed into the
magnetometer's \eeprom.

\section{\code{[controlsocket]}}

It is possible for the \code{send_cmd.py} program to send commands to
the magnetometer via the data recording daemon. Communication is via a
\udp\ socket or a unix domain socket. 

\subsection{\code{filename}}
Use a unix domain socket for communication with the data recording
daemon, with the given filename. If set to \code{none} then a
control socket will not be created. If the \code{filename} option is
present it takes priority over any \code{port} option.

Default: none.

\subsection{\code{port}}
Use a \udp\ socket for communication with the data recording
daemon, with the given port number. If set to \code{none} then a
control socket will not be created. If the \code{filename} option is
present it takes priority over the \code{port} option.

Default: \code{6587}

\configexample{\code{port = 6587}}

\section{\code{[magnetometer]}}

Settings associated with the magneometer.

\subsection{\code{siteid}}

The numeric site identifier for the magnetometer. The recording daemon
will ignore data packets where the site ID does not match the value
set in the configuration file. The site ID should be set as an integer
number in the range 0 to 255 inclusive.

Default: none.

\subsection{\code{key}}
The \hmac-\mdfive\ key used to sign communication messages sent between
the magnetometer and data recording daemon. This key should be kept
secret. The key is a 32 character hexadecial string, without any
\code{0x} prefix.

Default: none.\\
Typically set to a simple code, \code{key = 00000000000000000000000000000000}

\configexample{key = 65f024a22214ea2511ca60f251d7fb74}

