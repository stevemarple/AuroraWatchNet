\chapter{Configuration file options}

% \newcommand{\configexample}[1]{\examplebox{\code{#1}}}
\newcommand{\configexample}[2][]{\examplebox[#1]{#2}}

\section{Introduction}
Many of the AuroraWatchNet programs read a common configuration file,
typically located at \filename{/etc/awnet.ini}. The configuration file
is broken into sections, each of which starts with a section header in
square brackets (\code{[like_this]}). Other lines contain key names
and values, written as \code{key = value}. Leading and trailing
white space around both the key and value is ignored. Section headers
and key names do not contain white space, words may be separated with an
underscore. The configuration file can also contain comments, which
are entered with a hash (\code{\#}) or semi-colon (\code{;}) as the
first character.

The configuration file is parsed using Python's \code{SafeConfigParser}
module. This allows values defined in the same section, or in the
\code{[DEFAULT]} section to be inserted into other key value
definitions. This feature is commonly used to insert the site
abbreviation into filenames.

\section{\code{[DEFAULT]}}

The \code{[DEFAULT]} section is special as it defines values which can
used elsewhere in the configuration file.

\subsection{project}
Define the project. This is
used elsewhere within the configuration file, \eg, data filenames.

Default: none.

\subsection{site}
Define the site code, typically a three letter abbreviation. This is
used elsewhere within the configuration file, \eg, data filenames.

Default: none.

\configexample{\code{site = lan1}}

\section{\code{[awnettextdata]}}

Options associated with the standard text-format output data file.

\subsection{filename}
\label{sec:config-data-filename}
Define the filename used for text-format data files. This string is
expanded as a \code{strftime} format string and accepts the normal
\code{strftime} format specifiers; for a list of the acceptable format
specifiers see
\url{https://docs.python.org/2/library/time.html#time.strftime}. However,
since Python expands the string first any percent characters used as
part of a \code{strftime} format specifier must be repeated.

Default: none.

\configexample{
  \code{filename = /data/aurorawatchnet/\%(site)s/\%\%Y/\%\%m/\%(site)s_\%\%Y\%\%m\%\%d.txt}%
  \newline \newline % 
  \code{\%(site)} is replaced with the site abbreviation which was
  defined previously in the \code{[DEFAULT]} section. Notice how the
  \code{strftime} format specifiers require two \% characters.\newline
  \newline For June 20th 2014 with the site abbreviation
  \code{cwx} this would expand to\newline %
  \code{filename = /data/aurorawatchnet/cwx/2014/06/cwx_20140620.txt}
}

\section{\code{[awpacket]}}
Options associated with the standard binary output format. This format is
inconvenient to read but preserves the received data messages from the sensor
unit, and the responses sent back from the recording daemon. It is possible to
play back these files to the recording daemon and regenerate other data formats.

\subsection{filename}

See section~\ref{sec:config-data-filename} for a description.

Default: none.

\configexample{%
  Typically set to \\
  \code{filename =
    /data/aurorawatchnet/\%(site)s/\%\%Y/\%\%m/\%(site)s_\%\%Y\%\%m\%\%d.awp}
}

\subsection{key}
By default the binary data packets are written out with their original signing
key. If you plan to make the binary data format available then for security
reasons you should probably set a different signing key. You will then be able
to share this key without compromising the communication channel to the sensor
unit, and others will be able to use the error-checking capabilities provide by
\hmac-\mdfive. The key is a 32 character hexadecimal string, without any
\code{0x} prefix.

Default: none.

\configexample{
  Typically set to a simple code, \\
  \code{key = 00000000000000000000000000000000}
}

\section{\code{[logfile]}}
Options associated with the recorded log files. 

\subsection{filename}

See section~\ref{sec:config-data-filename} for a description.

Default: none.

\configexample{%
  Typically set to \\
  \code{filename =
    /data/aurorawatchnet/\%(site)s/\%\%Y/\%\%m/\%(site)s_\%\%Y\%\%m\%\%d.log}
}


\section{\code[daemon]}
Options relating to the data-recording daemon.

\subsection{\code{connection}}
Defines the connection between the daemon and the sensor unit. Radio
communication emulates a serial port connection. If the sensor unit is a \PoE\
model the communication should be set to \code{ethernet}.

Default: \code{serial}

\subsection{\code{close_after_write}}
Have the daemon close the data and log files after each write. It is
recommended that this option only be used on an \nfs\ file system,
when used on a file system store on flash memory (such as the Raspberry
Pi) this may cause excessive writes and early failure of the flash
memory. This option may be required for real-time data collection
daemons where the data files are also received by \code{rsync} or
\code{HTTP} uploads.

\subsection{\code{acknowledge}}
If set to \code{false} then acknowledgements are not sent in response
to sensor unit data messages.

Default: \code{true}

\subsection{\code{read_only}}
If set to \code{true} then for serial connections the device is opened
read-only and no set up commands are sent.

Read-only mode implies that acknowledgements are not sent, regardless
of the setting of the \code{acknowledge} option.

Default: \code{false}

\ifdef{\usesMicrocontroller}{%
  \section{\code{[serial]}}
  Options relating to serial data communication between the
  sensor unit and data recording daemon. This section is not used
  when communication uses Ethernet.

  \subsection{\code{port}}
  The filename of the serial port. For the Ciseco \emph{Slice of
    Radio} module this should be set to \filename{/dev/ttyAMA0}, for
  the Ciseco \emph{URF} module this will probably be
  \filename{/dev/ttyACM0}.

  Default: \filename{/dev/ttyACM0}

  \subsection{\code{baudrate}}
  Baud rate used with the serial port. For the Ciseco \emph{Slice of
    Radio} module this should be set to \code{9600}, for the Ciseco
  \emph{URF} module a higher rate of \code{57600} can be used.

  Default: \code{9600}

  \subsection{\code{setup}}
  The set-up string which should be sent to the serial device, for
  defining the channel number used for communication \etc.

  Default: none.

  Typically set to \code{ATRE;ATCN 25;ATLI R;ATAC}. Ensure that the
  channel number defined here matches the setting programmed into the
  sensor unit's \eeprom.

  \section{\code{[ethernet]}}

  These configuration options are used by the data recording daemon
  only when an Ethernet connection with the sensor unit is in use.

  \subsection{\code{local_address}}
  The local \ip\ address used for the data recording daemon. By
  default the daemon listens on all available interfaces.

  Default: empty string.

  \subsection{\code{local_port}}
  The port number of which the recording daemon should listen for sensor unit
  data packets.

  Default: \code{6588}

  \subsection{\code{remote_address}}
  The \ip\ address of the sensor unit.

  \subsection{\code{remote_port}}
  The port on which the sensor unit expects to receive acknowledgements.
}{}


\section{\code{[controlsocket]}}

It is possible for the \filename{send_cmd.py} program to send commands
to the sensor unit via the data recording daemon. Communication is
via a \udp\ socket or a Unix domain socket.

\subsection{\code{filename}}
Use a Unix domain socket for communication with the data recording
daemon, with the given filename. If set to \code{none} then a
control socket will not be created. If the \code{filename} option is
present it takes priority over any \code{port} option.

Default: none.

\subsection{\code{port}}
Use a \udp\ socket for communication with the data recording
daemon, with the given port number. If set to \code{none} then a
control socket will not be created. If the \code{filename} option is
present it takes priority over the \code{port} option.

Default: \code{6587}

\configexample{\code{port = 6587}}

\section{\code{[magnetometer]}}

Settings associated with the sensor unit.

\ifdef{\riometerManual}{%
  \helpbox{With the benefit of hindsight this section should have been called
    \code{sensorunit}. To maintain compatibility with existing installations the
    original name is retained, even when used in riometer systems.}
}{}

\subsection{\code{siteid}}

The numeric site identifier for the \instrumentType. The recording daemon
will ignore data packets where the site ID does not match the value
set in the configuration file. The site ID should be set as an integer
number in the range 0 to 65535 inclusive.

Default: none.

%% Magnetometer communication key only relevant for systems using
%% microcontroller.
\ifdef{\usesMicrocontroller}{%
  \subsection{\code{key}}
  \label{sec:config-file-hmac-key}
  The \hmac-\mdfive\ key used to sign communication messages sent between the
  sensor unit and data recording daemon. \textbf{This key should be kept
    secret.} The key is a 32 character hexadecimal string, without any \code{0x}
  prefix. The key used here must match that described in
  \secname~\ref{sec:eeprom-hmac-key}.

  Default: none.

  Set to a random value.
}{}

%% Firmware settings relevant only for systems using Calunium
%% microcontroller.
\ifdef{\usesMicrocontroller}{%
  \section{\code{[firmware]}}
  Configuration details for firmware upgrades.
  
  \subsection{\code{path}}
  The directory in which the firmware upgrades relevant to the
  \instrumentType\ hardware are stored. 
  
  \warningbox{It is critical that this path is set correctly, otherwise
    incompatible firmware upgrades could be delivered.}
}{}

\section{\code{[upload]}}
Configuration details for the \filename{upload_data.py} program.

\subsection{\code{method}}
Define the upload method in use. Valid options are: \code{rsync},
\code{rrsync}, \code{http}, and \code{https}. \code{rsync} uses the
\filename{rsync} program to efficiently transfer only the portions of
data files which have changed since the last upload; \code{rrsync} is
similar but the server restricts which directories may be written
to. Both are tunnelled through \ssh\ and thus require the network to
permit outward \tcp\ connections to port 22.

The \code{http} and \code{https} upload methods make use of the standard
\http(s) protocol, and therefore require outgoing \tcp\ connections to
port 80 and 443 respectively. 

\subsection{\code{rsync_host}}
Name of the \filename{rsync} host. No facility to set the user is
provided, use the \ssh\ \filename{config} file to set the
appropriate details.

Default: none.

\configexample[Example \protect\ssh\ \protect\filename{config} file:]{%
  \code{Host awn-data}\\ \code{Hostname host.domain.com} \\%
  \code{User monty}}

\subsection{\code{path}}
Name of the target directory used on the remote host for \code{rsync}
uploads. Ignored when the \code{method} is \code{http}, \code{https}
or \code{rrsync}. If not given the default path is constructed from
\filename{/data/aurorawatchnet/}, with the lower-case site
abbreviation appended as the final directory in the path.

\subsection{\code{url}}
The \URL\ to which \http/\slash \https\ uploads are sent.

\subsection{\code{password}}
The upload password for \http\ and \https\ methods. Digest
authentication is used.

\subsection{\code{username}}
The upload username for \http\ and \https\ methods. If not specified
it defaults to the site abbreviation prefixed with \code{awn-}.

\subsection{\code{realm}}
The \emph{realm} used for digest authentication when uploading data
with the \http\ or \https\ method.

\section{\code{[realtime_transfer]}}

The data recording daemon is capable of forwarding the incoming binary
packets to one or more remote hosts. When this is enabled data is
transferred in real-time. The remote host(s) can use the same data
recording daemon but should be configured to be read-only so that
acknowledgements are not sent.

When data packets are forwarded in this way the daemon does not check
that they have been delivered successfully. It is recommended that
this method is used in conjunction with the \filename{upload_data.py}
program to ensure any data packets which were not delivered are
transferred by some other means.

Real-time transfer uses three keys, \code{remote_host},
\code{remote_port} and \code{remote_key}. To send data to multiple
hosts a suffix is applied to each of this keys to group the settings
together, \eg, \code{remote_host2}, \code{remote_port2} and
\code{remote_key2}. The suffix must not contain white space characters.

\subsection{\code{remote_host}}
The hostname or \ip\ address to which data packets are sent.

\subsection{\code{remote_port}}
The \udp\ port to which data packets are sent.

\subsection{\code{remote_key}}
The \hmac-\mdfive\ key used to sign the data packets. This should be
different to the key used to communicate with the sensor unit.

\section{\code{[dataqualitymonitor]}}

The data recording daemon can monitor for the existence of a file
indicating that data quality may be compromised (\eg, due to local
site activities such as cutting grass). If the file is present an
extension is appended to the data files to clearly separate the poor
quality data. Any real time data transfers in operation are
suspended. When the file is removed the normal filenames are used
again and real time data transfer resume as normal.

\subsection{\code{port}}
Defines the serial port used by the \filename{awnetd_monitor.py}
daemon. If using a \usb\ device it is recommended to modify the
\code{udev} rules so that a fixed device name is generated based on
the serial number.

\helpbox{Example \code{udev} rule to create a symbolic link for an
  FTDI \usb\ serial adaptor, save to
  \filename{/etc/udev/rules.d/90-usb_serial.rules} or similar:\\%
  \\
  \small
  \# Have symlinks based on serial number for FTDI devices. Used for \\
  \# awnetd_monitor\\
  SUBSYSTEMS=="usb", KERNEL=="ttyUSB[0-9]*", \textbackslash \\
  ATTRS\{idVendor\}=="0403", ATTRS\{idProduct\}=="6001", SYMLINK+="tty_ftdi_\%s\{serial\}"}

Default: none.

\subsection{\code{filename}}
The name of the file which when present indicates reduced data
quality. This file can be created by the \filename{awnetd_monitor}
daemon or an external process.

Default: \filename{/var/aurorawatchnet/data_quality_warning}

\subsection{\code{extension}}
The extension which is appended to files to indicate reduced data
quality.

Default: \filename{.bad}

\subsection{\code{pidfile}}
A file containing the process ID of the \filename{awnetd_monitor}
daemon.

Default: none.

\subsection{\code{logfile}}
The file where the log output of the \filename{awnetd_monitor}
daemon is written.

Default: none.

\subsection{\code{led_active_low}}
A flag indicating if the \dtr\ signal line should be pulled low to
turn on the \led.

Default: true.

\section{\code{[ntp_status]}}

Configuration details for data recording daemon (\filename{awnetd.py})
and also used by the \filename{check_ntp_status} program. As the
Raspberry Pi does not have an onboard real-time clock it must obtain
the time from the network using \ntp. At boot time, and other
occasions when the \ntp\ servers are not accessible, the system clock
on the Raspberry Pi may not be correct. The data recording daemon can
be made aware that \ntp\ is running and correctly synchronized by the
presence of a semaphore file. The file's age is also checked to ensure
that the presence of a stale semaphore file is not
misinterpreted. When the data recording daemon is configured to check
\ntp\ status the current time is sent to the microcontroller only if
\ntp\ is synchronized.

\subsection{\code{filename}}
The filename of the semaphore file. Required for
\filename{check_ntp_status}. Not required by \filename{awnetd.py} and
if missing \ntp\ status is not checked.

\subsection{\code{max_age}}
The maximum allowable age in seconds for the semaphore file to be
considered valid. The presence of a file older than this age is
ignored. Required for \filename{check_ntp_status}. Not required by
\filename{awnetd.py} and if missing \ntp\ status is not checked.


